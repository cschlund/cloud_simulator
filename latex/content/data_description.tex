

\subsection{ERA-Interim reanalysis}\label{sec:era}

The ERA-Interim global atmospheric reanalysis \cite{Dee2011} provided by ECMWF 
(European Centre for Medium-Range Weather Forecasts) 
is the follow-up of ERA-40 reanalysis \cite{Uppala2005}, which was completed in 2002.
It covers the period from 1979 onwards and is continuously extended in near-real time.
One of the main objectives was to solve various difficulties regarding data assimiliation
(e.g. use of satellite data), which were found during the production of ERA-40. 
Evaluations of the new dataset have demonstrated that the hydrological cycle, the stratospheric circulation, and
the consistency in time of the reanalysis fields have been improved.
Data assimilation methodology, forecast model, and input observations
are essential components for the generation of high quality reanalysis data,
and consequently, will lead to more confident results in climate change applications. 

The reanalysis is produced by the Integrated Forecast System (IFS), which includes
the forecast model consisting of three fully coupled components for the atmosphere, land surface, and ocean waves.
ERA-Interim clouds are represented by a fully prognostic cloud scheme where cloud-related processes
are treated in a unified way, i.e. they are physically realistic and consistent with the rest of the model.
The scheme for stratiform and convective clouds has been developed by Tiedtke \cite{Tiedtke1993}, 
implemented in the global forecast model operationally in 1995 and since then, continuously improved.
An important and indispensable limitation of large-scale models is the fact that only bulk
properties of clouds can be taken into account. Hence, clouds are defined by
the horizontal coverage of the grid box by cloud, and the mass mixing ratio of total cloud condensate,
along with the constraint that cloud air is saturated with regard to water and ice, respectively.
The thermodynamic phase is determined on the basis of the temperature.
In other words, cloud cover and cloud water/ice content are derived from prognostic equations,
which follow the mass balance equation for cloud water/ice content and cloud air.
As time evolves in the simulation the cloud variables are changing due to source and
sink terms that are related to cloud formation (e.g., condensation/sublimation, cumulus convection)
and destruction (e.g., evaporation, precipitation) processes, respectively.

As already metioned above the original Tiedtke scheme has been continuously further developed
aiming at a more physically realistic representation of cloud and precipitation microphysics
that cannot be achieved by using just two prognostic cloud variables.
A revised Tiedtke cloud parameterization of stratiform cloud and precipitation became available by 2010
increasing the number of prognostic variables from two (cloud fraction, cloud condensate) to five
(cloud fraction, cloud liquid water, cloud ice, rain, and snow).
The updated approach used in ERA-Interim treats now water and ice clouds independent leading to more realistic
modelling of supercooled liquid water clouds as well as snow and rain precipitation processes.


% % % taken from Dee et al. 2011 % % %
% Improvements of the model physics have caused a positive impact on the 
% representation of clouds in the ERA-Interim dataset \cite{Dee2011}. By comparing the ERA-Interim and ERA-40 
% reanalysis with ISCCP observations several improvements regarding cloud cover have been identified. 
% For instance, the marine stratocumulus cloud cover has been increased by 15 -- 25~\%
% because a new moist boundary-layer scheme was implemented \cite{Koehler2011}.
% The tropical ocean total cloud cover has been decreased by 5 -- 15~\% 
% due to an overall improved hydrological cycle and the incorporation of ice supersaturation 
% delaying the formation of ice clouds \cite{Tompkins2007}.
% The cloud cover over land in the tropics has been increased by 20 -- 30~\% caused by an
% increase in high cloud resulting from improved deep convective triggering \cite{Bechtold2004} along
% with low cloud produced by a new boundary-layer scheme.
% The midlatitude ocean (low- and medium-height) cloud cover has been increased by about 5~\% due to
% improved numerics of the cloud scheme.



\subsection{ESA's Cloud\_cci climatology}\label{sec:cloudcci}
In 2010 the European Space Agency (ESA) has launched the Climate Change Initiative (CCI) project 
composed of several sub-projects providing essential climate variables (ECVs) from the
atmosphere, ocean, and land, respectively \cite{Hollmann2013}.
The principle purpose of CCI is the generation of satellite-based climate data records that
meet the challenging requirements of the Global Climate Observing System (GCOS) \cite{GCOS2011}.

The ESA Cloud\_cci sub-project is dedicated to the generation of long-term coherent A(A)TSR, AVHRR and MODIS 
cloud property datasets (1982 - 2014) along with associated uncertainties obtained by optimal estimation (OE) theory.
Emphasis is placed on the usage of synergetic capabilities of past, existing, and upcoming
European and US satellite missions (see Fig.~\ref{fig:cloudcci}) 
in order to improve accuracies as well as temporal and spatial sampling 
of the newly created climatology compared to others that are based on single sources \cite{Stengel2015} .
The OE based algorithm fits a physically consistent cloud/atmosphere/surface model to satellite observations 
simultaneously from the visible to the mid-infrared, thereby ensuring spectral consistency for 
all retrieved and derived parameters. This retrieval scheme, referred to as Community Optimal 
Estimation Cloud Retrieval for Climate (CC4CL), provides cloud fraction, 
cloud top level estimates (pressure, height, and temperature), cloud thermodynamic phase, 
cloud optical thickness, cloud effective radius and post processed products, such as cloud liquid and ice water path.
These cloud parameters are compared with the pseudo-satellite results derived from the cloud
simulator using ERA-Interim reanalysis as input (see Section~\ref{sec:sim_vs_obs}).

\begin{figure}[!t]
  \begin{minipage}[t]{0.5\textwidth}
%     \subfloat[ESA Cloud\_cci datasets along with equatorial crossing time (ECT).]
    {\includegraphics[scale=0.35]{./figures/{CC4CL_datasets}.png}\label{fig:cc4cl_input}}
  \end{minipage}
  \begin{minipage}[t]{0.5\textwidth}
%     \subfloat[ECT of AVHRR's on-board NOAA and MetOp platforms.]
    {\includegraphics[scale=0.21]{./figures/{Plot_AVHRR_equat_cross_time_19780101_20170101}.png}\label{fig:ahvrr_ect}}
  \end{minipage}
  \caption[ESA Cloud\_cci datasets along with equatorial crossing times.]
{The ESA Cloud\_cci project provides long-term coherent cloud property datasets using A(A)TSR-MODIS-AVHRR 
measurements from 1982 to 2014. The equatorial crossing time (ECT) of the satellites is shown in the left sketch 
(see table), while the ECTs of AVHRR bearing platforms are shown in the right figure.
The ECT of NOAA satellites changes in value over time due to orbit degradation, while the ECT of MetOp
platforms remain nearly constant throughout the year because they are in controlled orbits.}
\label{fig:cloudcci}
\end{figure}

