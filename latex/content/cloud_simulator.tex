
%\section{Description of the cloud simulator}\label{simpsimu}

This section describes the main steps of retrieving monthly mean 
Cloud\_cci-like (i.e. pseudo-satellite) observations derived from 
global atmospheric reanalysis produced by ECMWF.
The clouds simulator reads the 6-hourly (00, 06, 12, 18 UTC) 
ERA-Interim files per month containing gridbox mean vertical profiles of
\begin{itemize}
    \item pressure level $P_{lev}$,
    \item liquid water content $LWC$,
    \item ice water content $IWC$,
    \item cloud cover $CC$,
    \item geopotential height $h_{geo}$, and
    \item temperature $T$.
\end{itemize}
A detailed description of the ERA-Interim data is given in section~\ref{sec:era}.
The following steps address the things needed to ``simulate'' the observational process.
In other words, \textbf{what would a satellite see if the atmosphere had the clouds
of a climate model}?

%; means (in/out), temps (out)
%PSEUDO_RETRIEVAL, input, grid, sza2d, set.SCOPS, $ 
%    cwp_lay, cot_lay, cer_lay, set.COT, set.MPC, $
%    his, means, temps, TEST=test
%
%; sum up cloud parameters
%SUMUP_VARS, means, counts, temps

\subsection{Calculation of cloud effective radius, optical thickness and water path}
In the first step the cloud (liquid and ice) water content at each model level
is weighted by its cloud cover providing the so-called ``in-cloud'' cloud water content.
This cloud-cover-weighted product is then used for computing the ``in-cloud''
effective radius, optical thickness and water path for each model layer.

\subsubsection{Water droplet effective radius}
In ERA-Interim the liquid cloud effective radius is calculated following the method
described by Martin et al. (1994).

\subsubsection{Ice crystal effective radius}


\subsection{Pseudo retrieval}
\subsubsection{Statistical aggregation}

%\begin{figure}[!ht]
%  \begin{minipage}[c]{0.5\textwidth}
%    \includegraphics[scale=0.27]{./figures/{sst_era_interim_0.5_0.5}.png}
%  \end{minipage}\hfill
%  \begin{minipage}[c]{0.5\textwidth}
%    \includegraphics[scale=0.27]{./figures/{lsm_era_interim_0.5_0.5}.png}
%  \end{minipage}
%  \caption[ERA-Interim sea surface temperature and land-sea mask.]
%  {ERA-Interim sea surface temperature (left) and land-sea mask (right).} 
%  \label{fig:lsm}
%\end{figure}
%
%
%% CER, CWP, COT only daytime
%\begin{figure}[!ht]
%  \begin{minipage}[c]{0.5\textwidth}
%    \includegraphics[scale=0.27]{./figures/{\szamidnight}.png}
%    \includegraphics[scale=0.27]{./figures/{\szamidday}.png}
%  \end{minipage}\hfill
%  \begin{minipage}[c]{0.5\textwidth}
%    \includegraphics[scale=0.27]{./figures/{\szamorning}.png}
%    \includegraphics[scale=0.27]{./figures/{\szaevening}.png}
%  \end{minipage}
%  \caption[Solar zenith angle for 00, 06, 12, 18 UTC.]
%  {Solar zenith angle maps for 00, 06, 12, and 18 UTC on 1$^{st}$ of \MonthYear.} 
%  \label{fig:sza}
%\end{figure}
%
%
%% SCOPS snapshots 
%\begin{figure}[!ht]
%  \centering
%  \begin{minipage}[c]{0.4\textwidth}
%    \includegraphics[scale=0.20]{./figures/\snapscops_cloud_fraction1.png}
%    \includegraphics[scale=0.20]{./figures/\snapscops_cloud_phase1.png}
%    \includegraphics[scale=0.20]{./figures/\snapscops_cloud_effective_radius1.png}
%    \includegraphics[scale=0.20]{./figures/\snapscops_cloud_optical_thickness1.png}
%    \includegraphics[scale=0.20]{./figures/\snapscops_cloud_water_path1.png}
%  \end{minipage}
%  \begin{minipage}[c]{0.4\textwidth}
%    \includegraphics[scale=0.20]{./figures/\snapscops_cloud_fraction9.png}
%    \includegraphics[scale=0.20]{./figures/\snapscops_cloud_phase9.png}
%    \includegraphics[scale=0.20]{./figures/\snapscops_cloud_effective_radius9.png}
%    \includegraphics[scale=0.20]{./figures/\snapscops_cloud_optical_thickness9.png}
%    \includegraphics[scale=0.20]{./figures/\snapscops_cloud_water_path9.png}
%  \end{minipage}
%  \caption[Down-scaled cloud paramters.]
%  {These figures show the sub-column distribution of
%cloud fraction, phase, effective radius, optical thickness and water path 
%derived from the corresponding grid-box mean profiles. 
%The left and right columns demonstrates two different grid cells
%at 00 UTC on 1$^{st}$ of \MonthYear.}
%  \label{fig:scops}
%\end{figure}


