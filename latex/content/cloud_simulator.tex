
%\section{Description of the cloud simulator}\label{simpsimu}

This section describes the main steps of retrieving monthly mean 
Cloud\_cci-like (i.e. pseudo-satellite) observations derived from 
global atmospheric reanalysis produced by ECMWF.
The clouds simulator reads the 6-hourly (00, 06, 12, 18 UTC) 
ERA-Interim files per month containing gridbox mean vertical profiles of
\begin{itemize}
    \setlength\itemsep{0.2em}
    \item pressure level $P_{lev}$ [Pa],
    \item liquid water content $LWC$ [kg/kg]$\footnote{mass of condensate per mass of moist air}$,
    \item ice water content $IWC$ [kg/kg],
    \item cloud cover $CC$ (0-1),
    \item geopotential height $h_{geo}$ [m$^{2}/s^{2}$], and
    \item temperature $T$ [K].
\end{itemize}
A detailed description of the ERA-Interim data is given in section~\ref{sec:era}.
The following steps address the things needed to simulate the observational process.
In other words, what would a satellite see if the atmosphere had the clouds
of a climate model?


%\subsection{Calculation of cloud effective radius, optical thickness and water path}
\subsection{Pre-processing}

In the first step the gridbox mean liquid and ice water contents at each model level
are weighted by the cloud cover profile providing the so-called 
``in-cloud'' liquid and ice water contents. 
Thus, only the cloudy part of the grid cell is taken into account, 
which is comparable to what a satellite would detect.
Based on these cloud-cover-weighted profiles the liquid ($LWP$) and ice ($IWP$) water paths
per model layer can be derived, which present the total amount of liquid and ice
water between two consecutive pressure levels in the atmosphere, respectively.
The total amount of $LWP$ and $IWP$ is referred to as cloud water path ($CWP$).

The cloud optical thickness ($\tau$) per layer is obtained by the method of \citet{Han1994}
\begin{equation}\label{eq:han}
    \tau = \frac{3}{4} \frac{CWP \cdot  Q_{ext}}{r_{e} \cdot \rho}
\end{equation}
where $CWP$ represents either $LWP$ or $IWP$ depending on the thermodynamic phase.
$Q_{ext}$ denotes the extinction coefficient, which is assumed to be 2 for water and 2.1 for ice.
The density $\rho$ is set to 1 $g/m^{3}$ for water and 0.9167 $g/m^{3}$ for ice.
For the computation of effective radii ($r_{e}$) per layer the 
ERA-Interim parameterizations are implemented in the cloud simulator.

The cloud droplet effective radius ($r_{e}^{liq}$) follows the method of \citet{Martin1994} and 
is defined as function of liquid water content and cloud droplet number 
concentration, which in turn depends on the wind speed.
For practical reasons the simulator uses constant values for the number of 
cloud condensation nuclei over land (300) and sea (100).
Hence, a land-sea mask (Fig.~\ref{fig:lsm}) is required, 
which is obtained from ERA-Interim Sea Surface Temperature (Fig.~\ref{fig:sst}) 
data aggregated on a 0.5$^{\circ}$ latitude-longitude grid.
The ice crystal effective radius ($r_{e}^{ice}$) is parameterized as a function of 
temperature and ice water content based on \citet{Sun1999}, which has been
revised by \citet{Sun2001}.

In CC4CL the simultaneously retrieved $\tau^{liq}$ ($\tau^{ice}$) and 
$r_{e}^{liq}$ ($r_{e}^{ice}$) are combined utilizing the relationship given in 
Eq.~\ref{eq:han} for deriving $LWP$ ($IWP$) as post-processed product.
Thus, the cloud simulator adopts this retrieval characteristic of CC4CL by
using this equation for computing the 
cloud optical thickness per model layer since ERA-Interim provides solely 
the information on the cloud water content and cloud amount.

%\captionsetup[subfloat]{position=top}
\begin{figure}[!h]
  \begin{minipage}[t]{0.5\textwidth}
    \subfloat[ERA-Interim land-sea mask (LSM).]
    {\includegraphics[scale=0.26]{./figures/{lsm_era_interim_0.5_0.5}.png}\label{fig:lsm}}
  \end{minipage}
  \begin{minipage}[t]{0.5\textwidth}
    \subfloat[ERA-Interim sea surface temperature (SST).]
    {\includegraphics[scale=0.26]{./figures/{sst_era_interim_0.5_0.5}.png}\label{fig:sst}}
  \end{minipage}
  \caption[ERA-Interim land-sea mask and sea surface temperature (31/08/2015).]
{ERA-Interim auxiliary data provided on a 0.5$^{\circ}$ latitude-longitude grid.}\label{fig:lsm_sst}
\end{figure}


\subsection{Downscaler}
In the next step the simulator has to adress the mismatch in scale between 
that of a ERA-Interim model grid box ($\sim$ 500 km) and 
that of a satellite footprint ($\sim$ 5 km, e.g. AVHRR).
This is achieved by scaling the grid box mean profiles (e.g., 
Fig.~\ref{fig:profiles}) for each grid cell down to subcolumns$\footnote{
Currently each grid cell is broken into 20 subcolumns. This will be changed 
so that the number of subcolumns will be a function of latitude.}$, which
can be thought of as respresenting the spatial resolution of the instrument.
Figure~\ref{fig:scops} shows examples of vertical profiles of individual subcolumns 
for two different grid cells based on the grid box mean profiles displayed in 
Figure~\ref{fig:profiles}.
The upper left figure illustrates the subgrid cloud fraction profiles 
%The downscaler applies a pseudo-random sampling process 
%to generate an ensemble of subgrid profiles, which the distribution within the model grid box.

The cloud simulator assumes maximum random overlap of clouds, which means 
that clouds in neighboring layers are maximally overlapped, 
while groups of clouds separated by one or more clear layers are randomly overlapped.

%This method uses a pseudo-random sampling process to generate an ensemble of subgrid cloud
%profiles representing the distribution within the model grid box. It takes the input vertical
%profiles of cloud fraction to generate a specified number of horizontally homogeneous cloud
%profiles. The SCOPS algorithm splits each grid column into a number of subcolumns NCOL in which
%each layer is either completely filled or completely free of cloud. The cloud cover in each
%vertical layer and subcolumn is therefore either 0 or 1. The SCOPS version used here assumes
%maximum random overlap for cloud

The used approach for deriving subgrid profiles is very similar to
SCOPS (Subgrid Cloud Overlap Profile Sampler, \citet{Webb2001}),
which is implemented, for instance, in COSP (CFMIP Observation Simulator
Package, \cite{Bodas2011}).
COSP is an integrated satellite simulator, which has been developed by 
the Cloud Feedback Model Intercomparison Project (CFMIP) community.
It is capable to simulate observations of multiple active and passive
satellite instruments (e.g., CloudSat, Calipso, ISCCP, MISR, MODIS, RTTOV) 
and therefore, allows the quantitative evaluation of clouds, humidity,
and precipitation processes in diverse numerical models.



% grid box mean profiles 
\begin{figure}[!h]
  \centering
  \begin{minipage}{0.3\textwidth}
    \subfloat[Cloud fraction.]
    {\includegraphics[scale=0.17]{./figures/\snapscops_CFC9_profile.png}}
  \end{minipage}\hfill
  \begin{minipage}{0.3\textwidth}
    \subfloat[Cloud optical thickness.]
    {\includegraphics[scale=0.17]{./figures/\snapscops_COT9_profile.png}}
  \end{minipage}\hfill
  \begin{minipage}{0.3\textwidth}
    \subfloat[Cloud water path.]
    {\includegraphics[scale=0.17]{./figures/\snapscops_CWP9_profile.png}}
  \end{minipage}
  \caption[Grid box mean profiles.]{Example of cloud fraction (a),
cloud optical thickness (b) and cloud water path (c) 
grid box mean profiles. The grid cell is located in the daytime 
(solar zenith angle equal 17.7) at N$31^{\circ}30^{\prime}$ and 
E$164^{\circ}0^{\prime}$ at UTC 00 on 1$^{st}$ of \MonthYear.}
  \label{fig:profiles}
\end{figure}


% subcolumn cloud distribution 
\begin{figure}[!h]
  \centering

  \begin{minipage}{0.3\textwidth}
    \subfloat[Cloud fraction.]
    {\includegraphics[scale=0.17]{./figures/\snapscops_CFC9.png}}
  \end{minipage}\hfill
  \begin{minipage}{0.3\textwidth}
    \subfloat[Cloud optical thickness.]
    {\includegraphics[scale=0.17]{./figures/\snapscops_COT9.png}}
  \end{minipage}\hfill
  \begin{minipage}{0.3\textwidth}
    \subfloat[Cloud water path.]
    {\includegraphics[scale=0.17]{./figures/\snapscops_CWP9.png}}
  \end{minipage}\hfill

  \begin{minipage}{0.3\textwidth}
    \subfloat[Cloud effective radius.]
    {\includegraphics[scale=0.17]{./figures/\snapscops_CER9.png}}
  \end{minipage}\hfill
  \begin{minipage}{0.3\textwidth}
    \subfloat[Cloud phase.]
    {\includegraphics[scale=0.17]{./figures/\snapscops_CPH9.png}}
  \end{minipage}

  \caption[Down-scaled cloud parameter profiles.]{Top-down: 
Subcolumn distribution of cloud fraction, phase, effective radius, 
optical thickness and water path derived from grid box mean profiles
shown in Figure~\ref{fig:profiles} for
two different grid cells at UTC 00 on 1$^{st}$ of \MonthYear.}
  \label{fig:scops}
\end{figure}




%2. search for upper-most cloud and collect cloud parameters
\subsection{Pseudo-retrieval}
operate on subcolumns
%In the previous section the pre-processing of pseudo-satellite 
%$\tau$, $r_{e}$, and $CWP$ vertical profiles has been explained,
%which complement along with the original ERA-Interim 3D fields the required
%input data for the pseudo retrieval.
%This is the essential part of the cloud simulator responsible for producing 
%output comparable to Cloud\_cci data in three main steps.
%Scale COT greater than 100.
%Only solar COT, CWP, CER due to CC4CL.


% cloud top parameters 
\begin{figure}[!h]
  \begin{minipage}{\textwidth}
    \subfloat[Grid cell at N$29^{\circ}30^{\prime}$ and E$162^{\circ}30^{\prime}$.]
    {\includegraphics[width=\textwidth]{./figures/\snapscops_retrieved1.png}}
  \end{minipage}\vspace*{0.5cm}
  \begin{minipage}{\textwidth}
    \subfloat[Grid cell at N$31^{\circ}30^{\prime}$ and E$164^{\circ}0^{\prime}$.]
    {\includegraphics[width=\textwidth]{./figures/\snapscops_retrieved9.png}}
  \end{minipage}
  \caption[Retrieved subcolumn cloud top parameters.]
{Retrieved subcolumn cloud fraction (CFC), cloud top pressure (CTP),
cloud top temperature (CTT), cloud top height (CTH),
cloud phase (CPH), cloud effective radius (CER),
cloud optical thickness (COT), and cloud water path (CWP) 
for two different grid boxes at UTC 00 on 1$^{st}$ of \MonthYear.}
\label{fig:retrieved}
\end{figure}



%3. collect statistics, i.e. means over subcolumns and 1D, 2D histograms
\subsection{Compute summary statistics}


%% CER, CWP, COT only daytime
%\begin{figure}[!h]
%  \begin{minipage}[c]{0.5\textwidth}
%    \includegraphics[scale=0.27]{./figures/{\szamidnight}.png}
%    \includegraphics[scale=0.27]{./figures/{\szamidday}.png}
%  \end{minipage}\hfill
%  \begin{minipage}[c]{0.5\textwidth}
%    \includegraphics[scale=0.27]{./figures/{\szamorning}.png}
%    \includegraphics[scale=0.27]{./figures/{\szaevening}.png}
%  \end{minipage}
%  \caption[Solar zenith angle for 00, 06, 12, 18 UTC.]
%  {Solar zenith angle maps for 00, 06, 12, and 18 UTC on 1$^{st}$ of \MonthYear.} 
%  \label{fig:sza}
%\end{figure}
%
%


