
\section{Description of the cloud simulator}\label{simpsimu}

%The simplified simulator calculates global monthly means of Cloud\_cci-like (i.e.
%pseudo-satellite) parameters based on reanalysis data.
%In the case of evaluating ERA-Interim cloud parameters, it works as follows:
%\begin{enumerate}[label=(\alph*)]
%% \begin{enumerate}[label=(\Alph*)]
%% \begin{enumerate}[label=(\roman*)]
%
%
% \item\label{it:readdata} 
%The simulator starts with reading 6-hourly (00, 06, 12, 18 UTC)
%three-dimensional analysis fields of the
%\begin{itemize}
%	\item pressure level $P_{lev}$,
%	\item liquid water content $LWC$,
%	\item ice water content $IWC$,
%	\item cloud cover $CC$,
%	\item geopotential height $h_{geo}$, and
%	\item temperature $T$,
%\end{itemize}
%along with longitude $x$, latitude $y$, and pressure level $z$ information. 
%A detailed description of the ERA-Interim data is given
%in section~\ref{sec:era}.
%The following steps address the things needed to ``simulate'' the observational process.
%In other words, \textbf{what would a satellite see if the atmosphere had the clouds
%of a climate model}?
%
%
%%  \item\label{it:cwc_inc} In the next step a cloud-cover-weighted cloud water content
%% at each pressure level, $z$, is computed in the case of cloud occurrence
%% (i.e. $CC > 0$):
%% \begin{equation}\label{eq:incloud-cwc}
%%    CWC^{inc}(x,y,z) = \frac{CWC(x,y,z)}{CC(x,y,z)}
%% \end{equation}
%% with $inc$ denoting ``in-cloud'' and $CWC$ is either $LWC$ or $IWC$ providing
%% $LWC^{inc}$ or $IWC^{inc}$, respectively.
%
%
% \item\label{it:cwp_lay} 
%First of all, the \textbf{cloud water path for each layer} has to be calculated
%\begin{equation}\label{eq:cwp_lay}
%  CWP_{lay}(x,y,z') = CWC_{lay}(x,y,z') \cdotp \frac{\Delta P_{lev}(x,y,z')}{9.81}
%\end{equation}
%with $z'$ denoting the layer level ($z' = n_{z} - 1$; $n_{z}$ number of pressure levels),
%$\Delta P_{lev}$ being the pressure difference between two pressure levels, and
%the cloud water content per layer
%\begin{equation}\label{eq:cwc_lay}
%  CWC_{lay}(x,y,z') = \frac{1}{n} \sum\limits_{i=1}^n CWC(x,y,z_{i})\\[1.5ex]
%\end{equation}
%Depending on the thermodynamic phase, $CWC$ is either $LWC$ or $IWC$.
%Since a layer is defined as the region between two pressure levels, $n$ is equal 2.
%Thus, Eq.~\ref{eq:cwp_lay} provides the cloud liquid and ice water path for each
%layer, respectively.
%
%
%  \item\label{it:cot_inc_lay} 
%The next step aims at the computation of the
%\textbf{in-cloud cloud optical thickness per layer}.
%The word ``in-cloud'' refers to the part of the model grid cell, which is covered by clouds. 
%Therefore, a cloud-cover-weighted cloud water content at each pressure level is required
%\begin{equation}\label{eq:incloud-cwc}
%   CWC^{inc}(x,y,z) = \frac{CWC(x,y,z)}{CC(x,y,z)}
%\end{equation}
%because it is used to derive the in-cloud cloud water content per layer
%\begin{equation}\label{eq:cwc_inc_lay}
%  CWC^{inc}_{lay}(x,y,z') = \frac{1}{n} \sum\limits_{i=1}^n CWC^{inc}(x,y,z_{i})
%\end{equation}
%which in turn is used to calculate the in-cloud cloud water path for each layer
%\begin{equation}\label{eq:cwp_inc_lay}
%  CWP^{inc}_{lay}(x,y,z') = CWC^{inc}_{lay}(x,y,z') \cdotp \frac{\Delta P_{lev}(x,y,z')}{9.81}
%\end{equation}
%Eq.~\ref{eq:cwp_inc_lay} is then applied to derive the 
%in-cloud cloud optical thickness per layer, which is defined as
%\begin{equation}\label{eq:cot_inc_lay}
%  \tau^{inc}_{lay}(x,y,z') = \frac{3}{4} \frac{CWP^{inc}_{lay}(x,y,z') \cdotp Q_{ext}}{r_{eff} \cdotp \rho}
%\end{equation}
%following the same approach as used in CC4CL \cite{Han94}.
%Depending on the phase $CWP^{inc}_{lay}$ indicates either $LWP^{inc}_{lay}$ or
%$IWP^{inc}_{lay}$. $Q_{ext}$ denotes the extinction coefficient, which is assumed to be 2
%for water and 2.1 for ice. 
%\color{red}The cloud effective radius $r_{eff}$ is assumed to be 12 $\mu$m
%for water and 30 $\mu$m for ice, which are the a priori values used in CC4CL. \color{black}
%The density $\rho$ is 1 g m$^{-3}$ for water and 0.9167 g m$^{-3}$ for ice.
%Thus, Eq.~\ref{eq:cot_inc_lay} provides the in-cloud liquid and ice 
%cloud optical thickness for each layer, respectively.
%For the sake of convenience, $\tau^{inc}_{lay}$ will be referred to as $\tau$ from now on. 
%
% \item\label{it:search4cloud} 
%Afterwards the simulator iteratively searches bottom-up for the uppermost cloud layer 
%by means of a threshold approach applied 
%to the vertically integrated total in-cloud cloud optical thickness
%\begin{equation}
%\tau^{total}(x,y) = \sum\limits_{i=0}^{k} \tau^{liq}(x,y,z_{i}) + \tau^{ice}(x,y,z_{i})
%\end{equation}
%where $k$ is ranging from $(n_{z}-2)$ until 1.
%This means that in the first iteration $\tau^{total}$ represents 
%the vertical integration between the top of the model atmosphere
%and the last but two pressure level $0<i<n_{z}-2$. 
%In the next iteration, $k$ ranges from $i=0$ until $i=z-3$ and so on,
%until the last iteration considering the uppermost layer $0<i<1$ of the model atmosphere.
%If the given threshold is exceeded, the following cloud parameters are 
%aggregated on a 0.5 degree longitude-latitude grid:
%\color{red}
%\begin{center}
%\begin{tabular}{llll}
%\textbullet & Cloud top pressure & $CTP(x,y) = P_{lev}(x,y,z)$ & [hPa]\\[2.0ex]
%\textbullet & Cloud top height & $CTH(x,y) = h_{geo}(x,y,z)$ & [km]\\[2.0ex]
%\textbullet & Cloud top temperature & $CTT(x,y) = T(x,y,z)$ & [K] \\[2.0ex]
%\textbullet & Total column liquid water path &
%$LWP(x,y) = \sum\limits_{i=z_{0}}^z LWP_{lay}(x,y,i)$ & [g/m$^{2}$] \\[2.0ex]
%\textbullet & Total column ice water path &
%$IWP(x,y) = \sum\limits_{i=z_{0}}^z IWP_{lay}(x,y,i)$ & [g/m$^{2}$] \\[2.0ex]
%\textbullet & Cloud phase &
%$CPH(x,y) = \frac{LWP_{lay}(x,y,z)}{LWP_{lay}(x,y,z) +IWP_{lay}(x,y,z)}$ & [0;1]\\[2.0ex]
%\textbullet & Cloud fraction & $CFR(x,y) = max[CC(x,y,z_{0}:z)]$ & [0;1] \\
%\end{tabular}
%\end{center}
%
%In order to simulate the binary cloud mask (clear, cloudy) and cloud phase (liquid, ice)
%derived in the pre-processing of CC4CL (see Section~\ref{sec:cloudcci}), both the cloud
%fraction and cloud phase are rounded to its closest integer (i.e. 0 or 1) for each grid
%cell.\\
%Then, the cloud water path is reset depending on the cloud phase of the uppermost layer,
%i.e. if the cloud phase equals 0 (1), then the liquid (ice) water path is set to zero and
%the ice (liquid) water path is the sum of $LWP$ and $IWP$.
%This adaption is necessary because passive spaceborne instruments are not capable to
%retrieve information about the different levels within a cloud as compared to model
%data. Hence, the simulator has to take into account that a satellite sensor provides
%the vertical integrated liquid (ice) water content of existing cloud layers based on the
%thermodynamic phase of the uppermost cloud layer that could be identified by the retrieval
%system.
%
% \item\label{it:scaleCOT} Scale COT: max=100 and thus CWP
%
% \item\label{it:sunlitCOT} COT and CWP for sunlit region only
%
%\color{black}
%
%\end{enumerate}
%
%% Finally, 4 analyses per day per month are collected (i.e. iterating (a) $\rightarrow$ (e))
%% and summed up creating monthly gridded averages of original reanalysis (based on
%% thv$_{ori}$) and pseudo-satellite (based on thv$_{sat}$) cloud parameters stored in a
%% netCDF4 file.\\
%
%\textbf{Open issues}
%\begin{itemize}
%\item Cloud overlap (multi-layer clouds)?
%\end{itemize}
