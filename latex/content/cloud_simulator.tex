
%\section{Description of the cloud simulator}\label{simpsimu}

This section describes the main steps of retrieving monthly mean 
Cloud\_cci-like (i.e. pseudo-satellite) observations derived from 
global atmospheric reanalysis produced by ECMWF.
The clouds simulator reads the 6-hourly (00, 06, 12, 18 UTC) 
ERA-Interim files per month containing gridbox mean vertical profiles of
\begin{itemize}
    \setlength\itemsep{0.2em}
    \item pressure level $P_{lev}$ [Pa],
    \item liquid water content $LWC$ [kg/kg]$\footnote{mass of condensate per mass of moist air}$,
    \item ice water content $IWC$ [kg/kg],
    \item cloud cover $CC$ (0-1),
    \item geopotential height $h_{geo}$ [m$^{2}/s^{2}$], and
    \item temperature $T$ [K].
\end{itemize}
A detailed description of the ERA-Interim data is given in section~\ref{sec:era}.
The following steps address the things needed to ``simulate'' the observational process.
In other words, \textbf{what would a satellite see if the atmosphere had the clouds
of a climate model}?


\subsection{Calculation of cloud effective radius, optical thickness and water path}

First of all, the gridbox mean liquid and ice water contents at each model level
are weighted by the cloud cover profile providing the so-called ``in-cloud'' 
liquid and ice water contents in order to consider only the cloudy part of the grid cell.
Based on these cloud-cover-weighted profiles the liquid ($LWP$) and ice ($IWP$) water paths
per model layer can be derived, which present the total amount of liquid and ice
water between two consecutive pressure levels in the atmosphere, respectively.
The total amount of $LWP$ and $IWP$ is referred to as cloud water path ($CWP$).

The cloud optical thickness ($\tau$) per layer is obtained by the method of \citet{Han1994}
\begin{equation}\label{eq:han}
    \tau = \frac{3}{4} \frac{CWP \cdot  Q_{ext}}{r_{e} \cdot \rho}
\end{equation}
which is also used in CC4CL but to obtain the post-processed $CWP$ 
based on the retrieved optical thickness and effective radius.
$Q_{ext}$ denotes the extinction coefficient, which is assumed to be
2 for water and 2.1 for ice.
The density $\rho$ is set to 1 $g/m^{3}$ for water and 0.9167 $g/m^{3}$ for ice.
For the computation of effective radii $r_{e}$ per layer the 
ERA-Interim parameterizations are implemented in the cloud simulator.
The cloud droplet effective radius follows the method by \citet{Martin1994} and 
is defined as function of liquid water content and cloud droplet number 
concentration$\footnote{In the cloud simulator the number of cloud condensation nuclei
is set to constant values, i.e. 300 over land and 100 over sea, respectively.}$, 
which in turn depends on the wind speed.
% Dr Richard Forbes ---------------------------------------------------------
%ZNTOT = cloud droplet number concentration (dependent on wind speed in 
%ERAI parametrization, but you could choose fixed values for land and ocean
%e.g. (300 and 100 respectively)  
%----------------------------------------------------------------------------
The ice crystal effective radius is parameterized as a function of 
temperature and ice water content based on \citet{Sun1999}, which has been
revised by \citet{Sun2001}.


\subsection{Pseudo retrieval}
Addressing the mismatch between gridbox mean and satellite footprint;
``down-scaler'' 

\subsubsection{Statistical aggregation}
histograms and gridbox mean values

%\begin{figure}[!ht]
%  \begin{minipage}[c]{0.5\textwidth}
%    \includegraphics[scale=0.27]{./figures/{sst_era_interim_0.5_0.5}.png}
%  \end{minipage}\hfill
%  \begin{minipage}[c]{0.5\textwidth}
%    \includegraphics[scale=0.27]{./figures/{lsm_era_interim_0.5_0.5}.png}
%  \end{minipage}
%  \caption[ERA-Interim sea surface temperature and land-sea mask.]
%  {ERA-Interim sea surface temperature (left) and land-sea mask (right).} 
%  \label{fig:lsm}
%\end{figure}
%
%
%% CER, CWP, COT only daytime
%\begin{figure}[!ht]
%  \begin{minipage}[c]{0.5\textwidth}
%    \includegraphics[scale=0.27]{./figures/{\szamidnight}.png}
%    \includegraphics[scale=0.27]{./figures/{\szamidday}.png}
%  \end{minipage}\hfill
%  \begin{minipage}[c]{0.5\textwidth}
%    \includegraphics[scale=0.27]{./figures/{\szamorning}.png}
%    \includegraphics[scale=0.27]{./figures/{\szaevening}.png}
%  \end{minipage}
%  \caption[Solar zenith angle for 00, 06, 12, 18 UTC.]
%  {Solar zenith angle maps for 00, 06, 12, and 18 UTC on 1$^{st}$ of \MonthYear.} 
%  \label{fig:sza}
%\end{figure}
%
%
%% SCOPS snapshots 
%\begin{figure}[!ht]
%  \centering
%  \begin{minipage}[c]{0.4\textwidth}
%    \includegraphics[scale=0.20]{./figures/\snapscops_cloud_fraction1.png}
%    \includegraphics[scale=0.20]{./figures/\snapscops_cloud_phase1.png}
%    \includegraphics[scale=0.20]{./figures/\snapscops_cloud_effective_radius1.png}
%    \includegraphics[scale=0.20]{./figures/\snapscops_cloud_optical_thickness1.png}
%    \includegraphics[scale=0.20]{./figures/\snapscops_cloud_water_path1.png}
%  \end{minipage}
%  \begin{minipage}[c]{0.4\textwidth}
%    \includegraphics[scale=0.20]{./figures/\snapscops_cloud_fraction9.png}
%    \includegraphics[scale=0.20]{./figures/\snapscops_cloud_phase9.png}
%    \includegraphics[scale=0.20]{./figures/\snapscops_cloud_effective_radius9.png}
%    \includegraphics[scale=0.20]{./figures/\snapscops_cloud_optical_thickness9.png}
%    \includegraphics[scale=0.20]{./figures/\snapscops_cloud_water_path9.png}
%  \end{minipage}
%  \caption[Down-scaled cloud paramters.]
%  {These figures show the sub-column distribution of
%cloud fraction (0-1), phase (0-1), effective radius [$\mu$m], optical thickness 
%and water path [kg/m$^{2}$] derived from the corresponding grid-box mean profiles. 
%The left and right columns demonstrates two different grid cells
%at 0 UTC on 1$^{st}$ of \MonthYear.}
%  \label{fig:scops}
%\end{figure}


